\documentclass[a4paper,12pt,twoside]{article}
\usepackage[polish]{babel}
%Times New Roman
\usepackage{pslatex}
%graphics
\usepackage{graphicx}
\graphicspath{{./graphics/}}
\usepackage{caption}
\usepackage{subcaption}
\usepackage{float}
\usepackage{courier}

\usepackage[T1]{fontenc}
\usepackage[utf8]{inputenc}
\usepackage{times}
%wrapping figues
%\usepackage{wrapfig}
%\fontsize{5cm}{1em}


%margins
\usepackage[top=2cm, bottom=3cm]{geometry}
\setlength{\hoffset}{-1in}
\setlength{\oddsidemargin}{3cm}
\setlength{\evensidemargin}{2.5cm}
\setlength{\parskip}{6pt}

%spaces between paragraph
\usepackage{parskip}
\setlength{\parindent}{0pt}

%line spacing - 1,5
\linespread{1.5}

%font size of sections
%\usepackage{titlesec}
%\titleformat{\section}{\fontsize{20}{20}\bfseries}{\thesection}{1em}{}
%\titleformat{\subsection}{\fontsize{18}{18}\bfseries}{\thesubsection}{1em}{}
%\titleformat{\subsubsection}{\fontsize{16}{16}\bfseries}{\thesubsubsection}{1em}{}
%\titleformat{\paragraph}{\fontsize{14}{14}\bfseries}{\theparagraph}{1em}{}

\setcounter{tocdepth}{4}

%%% Number down to subsubsections only
\setcounter{secnumdepth}{3}

%awesome headers
\usepackage{fancyhdr}
\pagestyle{fancy}
\lhead{\fontsize{10pt}{1em}Justyna Kałużka\\ \em Aplikacja internetowa do ewidencji sprzetu w budynku Centrum Technologii Informatycznych}
\rhead{\fontsize{14pt}{1em}\slshape\bfseries\thepage}
\cfoot{}

%hyperlinks
\usepackage{hyperref}

%hilight yellow
\usepackage{color}
\newcommand{\hilight}[1]{\colorbox{yellow}{#1}}

\usepackage{listings}

\title{Aplikacja internetowa do ewidencji sprzetu w budynku Centrum Technologii Informatycznych}
\author{Justyna Kałużka\\dr inż. Grzegorz Jabłoński\\mgr inż. Piotr Nowak}
\date{\today}


\begin{document}

%\maketitle

\begin{titlepage}

\begin{center}
POLITECHNIKA ŁÓDZKA\\
Wydział Elektrotechniki, Elektroniki, \\
Informatyki i Automatyki\\
\vfill
Praca dyplomowa inżynierska\\
\textbf{Aplikacja internetowa do ewidencji sprzętu\\ w budynku Centrum Technologii Informatycznych\\
Justyna Kałużka}\\
\vfill
Nr albumu:  180194
\end{center}


\vfill
\hfill Opiekun pracy:

\hfill dr inż. Grzegorz Jabłoński

\vfill

\hfill Dodatkowy opiekun:

\hfill mgr inż. Piotr Nowak

\vfill
\begin{center}
Łódź, 2016
\end{center}

\end{titlepage}


\newpage
%--------------------ABSTRACT--------------------
\pagenumbering{gobble}
\addcontentsline{toc}{section}{Abstract}
\begin{center}
LODZ UNIVERSITY OF TECHNOLOGY\\
FACULTY OF ELECTRICAL, ELECTRONIC,\\
COMPUTER AND CONRTOL ENGINEERING\\
\vspace{1.5cm}
\textbf{Justyna Kałużka}\\
BSc Thesis\\
\textbf{Internet application for inventory management of laboratory equipment in Information Technology Centre}\\
\L \'od\'z, 2016
\end{center}
\hspace{1.15cm}Supervisors:\\
\vspace{0.5cm}
\hspace{1cm}Grzegorz Jabłoński, PhD Piotr Nowak,?
\vspace{1.5cm}
\begin{abstract}
blabla
\end{abstract}

%\newpage\null\thispagestyle{empty}\newpage
\newpage
\begin{center}
POLITECHNIKA ŁÓDZKA\\
WYDZIAŁ ELEKTROTECHNIKI, ELEKTRTONIKI\\
INFORMATYKI I AUTOMATYKI\\
\vspace{1.5cm}
\textbf{Justyna Kałużka}\\
Praca dyplomowa inżynierska\\
\textbf{Aplikacja internetowa do ewidencji sprzętu w budynku Centrum Technologii Informatycznych}\\
Łódź, 2016
\end{center}
\hspace{1.15cm}Opiekunowie pracy:\\
\vspace{0.5cm}
\hspace{1cm} dr inż. Grzegorz Jabłoński, mgr inż. Piotr Nowak
\vspace{1.5cm}

\renewcommand{\abstractname}{Streszczenie}
\begin{abstract}
Gnjsdds
\end{abstract}



\newpage
\setcounter{page}{1}
\pagenumbering{arabic}
\tableofcontents


\newpage
%--------------------INTRODUCTION--------------------
\section{Wstęp}
%\section*{Introduction}
%\addcontentsline{toc}{section}{Introduction}

This is the final report presenting the results of work on bachelor thesis.

\subsection{Motywacja?}


\section{Istniejące rozwiązania}


\section{Założenia projektu i wybor technologii}
odbiorcy, use casy
serwer aplikacyjny

-protokoły dla sali
-protokół ogólny
-wyszukiwarka wielu obiektów

\section{Implementacja}

\subsection{Użyte technologie}

Django\\
HTML, CSS, Bootstrap\\
bower\\
postgres

\paragraph{Rozszerzenia i biblioteki}~\
django-bootstrap3 - dlaczego nie?\\
pipeline - problemy z django 1.9\\
registeation - jak wyżej, do rozwiązania (grudzień 2015)\\
https://github.com/macdhuibh/django-registration-templates\\
django-extensions - awesome shell_plus\\
pep8 !!!!!!!!!!!\\

\subsection{Środowisko}
PyCharm na licencji studenckiej, virtualenvwrapper, 

\subsection{Struktura projektu}

domyślna struktura Django, przeniesione templaty

\subsection{Baza danych}

\section{Wdrożenie, }
jak działa, testy

\section{Podsumowanie}

http://axiacore.com/blog/effective-dependency-management-django-using-bower/
https://www.youtube.com/watch?v=qkFWkOw-ByU

\newpage

wymagania prawnee, inwentazyacja
wzorce projektowe
architektura plikacji webowych


\newpage
\addcontentsline{toc}{section}{List of Figures}
\listoffigures

\newpage
\raggedright
\begin{thebibliography}{9}
\addcontentsline{toc}{section}{References}
%\cite{frontback}


\bibitem{gcm}
Google Inc., 'Google Cloud Messaging for Android', \url{http://developer.android.com/google/gcm/index.html}, 4 February 2014, (accessed 5 February 2014).

\bibitem{arstechnica}
Amadeo, Ron, 'Google's iron grip on Android: Controlling open source by any means necessary', \url{http://arstechnica.com/gadgets/2013/10/googles-iron-grip-on-android-controlling-open-source-by-any-means-necessary/}, 21 October 2014, (accessed 8 February 2014).


\end{thebibliography}
\url{http://guides.is.uwa.edu.au/content.php?pid=385139&sid=3156563}

\end{document}
